\documentclass[11pt]{article}

\usepackage{amsfonts}
\usepackage{amsmath}
\usepackage{graphicx}
\usepackage{subfig}
\usepackage{float}
\usepackage{longtable}

\usepackage[margin=1in, paperwidth=8.5in, paperheight=11in]{geometry}
\usepackage[T1]{fontenc}
\usepackage[polish]{babel}
\usepackage[utf8]{inputenc}
\usepackage{lmodern}
\selectlanguage{polish}
\usepackage{pgf,tikz}
\usepackage{mathrsfs}
\usetikzlibrary{arrows}
\pagestyle{empty}

\begin{document}
	
	\title{Sprawozdanie 6 \\ \textbf{Zagadnienie początkowe dla układu równań różniczkowych zwyczajnych.}}
	\author{Dawid Stelmach}
	\date{\today}
	\maketitle\textsl{}
	
	\section{Treść 7}
	Metoda Rungego-Kutty rzędu 4-go (wzór "3/8") dla układu dwóch równań.
	
	\section{Metoda}
	
	\section{Algorytm}

	\section{Warunek STOPu}
	
	\section{Przykłady}
	
\end{document}