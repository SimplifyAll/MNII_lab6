\documentclass[11pt]{article}

\usepackage{amsfonts}
\usepackage{amsmath}
\usepackage{graphicx}
\usepackage{subfig}
\usepackage{float}
\usepackage{longtable}

\usepackage[margin=1in, paperwidth=8.5in, paperheight=11in]{geometry}
\usepackage[T1]{fontenc}
\usepackage[polish]{babel}
\usepackage[utf8]{inputenc}
\usepackage{lmodern}
\selectlanguage{polish}
\usepackage{pgf,tikz}
\usepackage{mathrsfs}
\usetikzlibrary{arrows}
\pagestyle{empty}

\begin{document}
	
	\title{Sprawozdanie 6 \\ \textbf{Zagadnienie początkowe dla układu równań różniczkowych zwyczajnych.}}
	\author{Dawid Stelmach}
	\date{\today}
	\maketitle\textsl{}
	
	\section{Treść 7}
	Metoda Rungego-Kutty rzędu 4-go (wzór "3/8") dla układu dwóch równań.
	
	\section{Metoda}
	Celem powyższego zadania jest rozwiązanie układu dwóch równań różniczkowych zwyczajnych z zagadnieniem początkowym Cauchy'ego w postaci:
	$$\begin{cases}
	y_1' = f_1(x,y_1,y_2) \\
	y_2' = f_2(x,y_1,y_2) \\
	y_1(a) = y_{1a} \\
	y_2(a) = y_{2a}
	\end{cases}
	$$
	W tym celu zastosowana zostanie metoda Rungego-Kutty czędu 4-go.
	\medskip
	Powyższa metoda należy do grupy jawnych i jednokrokowych tzn. do obliczenia wartości w węźle $y_{i_1}$ wykorzystujemy tylko wartość w węźle $y_i$.
	
	\medskip
	Aby rozwiązać równanie różniczkowe zwyczajne z zagadnieniem Cauchy'ego tzn:
	$$\begin{cases}
	y' = f(x,y) \\
	y(a) = y_{a}
	\end{cases}
	$$
	gdzie:
	\begin{itemize}
		\item $x \in [a,b]$ - przedział jest zdefiniowany dla danego problemu
		\item $y_a$ - przybliżenie początkowe $y(a)$, w punkcie będącym początkiem przedziału, przyjmujemy że $x_0 = a$
	\end{itemize}
	należy dla wszystkich węzłów wyznaczyć przybliżenia funkcji ze wzoru:
	$$ y_{i+1} = y_{i} + h*\displaystyle\sum_{j=1}^{r}c_jk_{j} $$
	gdzie:
	\begin{itemize}
		\item $y_i$ - przybliżenie $y(x_i)$
		\item $h = \frac{b-a}{n}$ - odległość między węzłami, gdzie:
		\begin{itemize}
			\item $a,b$ - punkty będące końcami rozważanego przedziału na którym funkcja jest określona
			\item $n$ - ilość podziałów powyższego przedziału
		\end{itemize}
		\item $r$ - rząd metody, w rozważanym przypadku $r = 4$
		\item $c_j$ - $j$-ta wartość z wektora $c$, będącego wektorem współczynników specyficznych dla danego wzoru Rungego-Kutty. W przypadku wzoru $(3/8)$:
		$$c=[asdf fsdaf sdaf saf saf sdf sdf grerag asdf ]$$
		\item $k_j$ - $j$-ta wartość z wektora $k$ obliczanego dla każdego węzła wg wzoru:
		$$ 	k_{j}=f(x_i+h\displaystyle\sum_{l=1}^{j-1}b_{j,l} \quad;\quad y_{i} + h\displaystyle\sum_{l=1}^{j-1}b_{j,l}k_{l} )$$
		gdzie:
		\begin{itemize}
			\item $b_{j,l}$ - $j,l$-ta wartość z macierzy $b$, będącej macierzą współczynników specyficznych dla danego wzoru Rungego-Kutty. W przypadku wzoru $(3/8)$:
			$$b=\left[ \begin{array}{cccc}
			<g_1, g_1> & <g_1, g_2> & <g_1, g_n> &  <g_1, g_n>\\
			<g_2, g_1> & <g_2, g_2> & <g_1, g_n> & <g_2, g_n> \\
			<g_1, g_n> & <g_1, g_n> & <g_1, g_n> & <g_1, g_n> \\
			<g_n, g_1> & <g_n, g_2> & <g_1, g_n> & <g_n, g_n> \\
			\end{array}\right]$$
			\item $k_{l}$ - wcześniej obliczone wartości wektora współczynników $k$
		\end{itemize}
	\end{itemize}
	
	W przypadku układu równań rozwiązanie w wykorzystaniem powyższych wzorów przyjmuje postać:
	$$\begin{cases}
	y_{1,i+1} = y_{1,i} + h*\displaystyle\sum_{j=1}^{r}c_jk_{1,j} \\
	
	y_{2,i+1} = y_{2,i} + h*\displaystyle\sum_{j=1}^{r}c_jk_{2,j} \\
	
	k_{m,j}=f_m(x_i+h\displaystyle\sum_{l=1}^{j-1}b_{j,l} \quad;\quad y_{1,i} + h\displaystyle\sum_{l=1}^{j-1}b_{j,l}k_{1,l} \quad;\quad y_{2,i} + h\displaystyle\sum_{l=1}^{j-1}b_{j,l}k_{2,l})
	\end{cases}
	$$
	\section{Algorytm}
	Kolejność wykonywania zadań:
	\begin{enumerate} 
		\item for i = 1 : n
		\begin{itemize}
			\item for j = 1 : r
			\begin{itemize}
				\item wyznacz $\displaystyle\sum_{l=1}^{j-1}b_{j,l}$
				\item wyznacz $\displaystyle\sum_{l=1}^{j-1}b_{j,l}k_{1,l}$
				\item wyznacz $\displaystyle\sum_{l=1}^{j-1}b_{j,l}k_{2,l}$
				\item oblicz $k_{1,j}$ oraz $k_{2,j}$
			\end{itemize}
			\item wyznacz $\displaystyle\sum_{j=1}^{r}c_jk_{1,j}$ oraz $\displaystyle\sum_{j=1}^{r}c_jk_{2,j}$
			\item oblicz $y_{1,i+1}$ oraz $y_{2,i+1}$
		\end{itemize}
	\end{enumerate}
	Powyższy algorytm zrealizowany jest w jednej funkcji:
	$$[X, Y1, Y2] = eqSet_runkut( f1, f2, y1_0, y2_0, a, b, n )$$
	przyjmującej jako parametry:
	\begin{itemize}
		\item $f1$ - pierwsza funkcja układu równań
		\item $f1$ - druga funkcja układu równań
		\item $y1_0$ - przybliżenie początkowe dla funkcji $f1$ w punkcie $a$
		\item $y2_0$ - przybliżenie początkowe dla funkcji $f2$ w punkcie $a$
		\item $a$ - początek przedziału
		\item $b$ - koniec przedziału
		\item $n$ - ilość podziałów rozważanego przedziału
	\end{itemize}
	Funkcja zwraca wartość parametru $\beta \in beta\_range$ dla którego błąd jest minimalny.
	
	\section{Przykłady}
	
\end{document}